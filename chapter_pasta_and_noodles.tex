\chapter{Pasta and Noodles}
\section{Eggless Pasta - Spaghetti, Lasagne, Bigoli, Vermicelli}
My italian friend once told me "What are you doing?! There is no egg in
spaghetti!". And I took that personally - there are no eggs in my spaghetti
anymore. As surprising as it may be, most traditional italian pasta do not
contain eggs - they are made of flour and water.  However, there is nothing
stopping anyone from using eggs in spaghetti. It simply will not be
"traditional". If you wish to use eggs, follow \fullref{egg-pasta}.

The italian eggless pasta are very similar to the japanese \fullref{udon}.

\header{Ingredients}
\begin{itemize}
  \item wheat flour
  \item 40-45\% water
\end{itemize}

\header{Instructions}
All the different types of pasta follow the same dough. The difference comes
from the shape, thickness and width of the final cut product. To produce round
pasta, you need an attachment for a pasta machine. More complex shapes
require a pasta extruder.
\begin{enumerate}
  \item Mix flour and water, then knead until no more flour remains.
  \item Leave to autolyse for 30 minutes to an hour.
  \saveenum
\end{enumerate}
If you own a pasta machine:
\begin{enumerate}
  \contenum
  \item Roll the dough out using a rolling pin until it fits in the machine.
    Keep the dough mostly rectangular.
  \item Run the dough through the machine several times until your preferred
    thickness.
  \item Cut the sheet of dough into a desired length sheets. Sprinkle the
    sheets with flour to prevent sticking.
  \item Using an attachment, cut the sheets into pasta.
\end{enumerate}
If you do not own a pasta machine:
\begin{enumerate}
  \contenum
  \item Roll the dough out using a rolling pin until your preferred thickness.
    Keep the sheet rectangular.
  \item Sprinkle the sheet genereously with flour, then fold the dough along
    the longer edge once or twice, depending on the size.
  \item Cut the dough with a sharp knife along the shorter edge into roughly
    equally sized stripes.
\end{enumerate}
The pasta is ready to be boiled or dried. Fresh pasta is good for about half a
day. Dried pasta may be stored for months.

\section{Egg Pasta - Fettuccine, Pappardelle, Tagliatelle, Tagliolini} \label{egg-pasta}

\section{Udon} \label{udon}
\header{Ingredients}
\begin{enumerate}
  \item wheat flour
  \item water 40\%
  \item salt 8\%
\end{enumerate}

\header{Instructions}
Mix and knead the ingredients into a rough dough. Cover and let rest for at
least 30 minutes.

Once ready to make the noodles, divide into 150g balls. Using a rolling pin,
roll a ball out to a thick sheet. Then, using a pasta machine, roll it to 3mm
thickness (number 2 on Atlas 150) gradually decreasing the thickness setting on
the machine. Fold in half, roll to 3mm thickness. Fold in half from the
opposite end and roll to 3mm thickness. At this point the dough should be
roughly in a rectangular shape. If it is not, continue folding and rolling. You
may fold from both ends at the same time overlapping the folds or turn the
dough 90 degrees and continue from there. Although udon is standardised to be
at least 1.7mm in diameter (when round) or width (when square), there is no one
true udon thickness, hence roll it to your preferences.

Once done, dust generously with flour. If you have a bigoli (3.5mm round),
trenette (3.5mm square), spaghetti (2mm round) or spaghetti chitarra (2mm
square) attachment, run your sheet through. Otherwise, place your sheet on a
cutting board and fold in half twice ensuring that the layers are generously
dusted with flour. Using a sharp knife cut using little pressure and moving
your knife back and forth. Do not press as that might cause the noodle to stick
to itself. Move the cut noodle to the side enrusing it does not stick to the
other noodles (the freshly cut sides are sticky).

Boil water, at least 1L as the boiling process is long. Once the water is
boiling, drop your noodles in, stir after 1 minute and then occasionally every
few minutes. Boiling udon takes at least 10 minutes, but it heavily depends on
the thickness. Thicker noodles require longer boiling.

Strain your noodles and cool them with cold water or in an ice bath.

\section{Soba}
