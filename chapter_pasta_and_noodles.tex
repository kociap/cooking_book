\chapter{Pasta and Noodles}
\section{Udon}
\header{Ingredients}
\begin{enumerate}
  \item wheat flour
  \item water 40\%
  \item salt 3-4\%
\end{enumerate}

For a single portion, 100g of flour is sufficient.

\header{Instructions}
Mix and knead the ingredients into a rough dough. Cover and let rest for at
least 30 minutes.

Once ready to make the noodles, divide into 150g balls. Using a rolling pin,
roll a ball out to a thick sheet. Then, using a pasta machine, roll it to 3mm
thickness (number 2 on Atlas 150) gradually decreasing the thickness setting on
the machine. Fold in half, roll to 3mm thickness. Fold in half from the
opposite end and roll to 3mm thickness. At this point the dough should be
roughly in a rectangular shape. If it is not, continue folding and rolling. You
may fold from both ends at the same time overlapping the folds or turn the
dough 90 degrees and continue from there. Although udon is standardised to be
at least 1.7mm in diameter (when round) or width (when square), there is no one
true udon thickness, hence roll it to your preferences.

Once done, dust generously with flour. If you have a bigoli (3.5mm round),
trenette (3.5mm square), spaghetti (2mm round) or spaghetti chitarra (2mm
square) attachment, run your sheet through. Otherwise, place your sheet on a
cutting board and fold in half twice ensuring that the layers are generously
dusted with flour. Using a sharp knife cut using little pressure and moving
your knife back and forth. Do not press as that might cause the noodle to stick
to itself. Move the cut noodle to the side enrusing it does not stick to the
other noodles (the freshly cut sides are sticky).

Boil water, at least 1L as the boiling process is long. Once the water is
boiling, drop your noodles in, stir after 1 minute and then occasionally every
few minutes. Boiling udon takes at least 10 minutes, but it heavily depends on
the thickness. Thicker noodles require longer boiling.

Strain your noodles and cool them with cold water or in an ice bath.

\section{Soba}
