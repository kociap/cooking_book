\chapter{Main Courses}
\section{Chicken Broccoli}
\section{Katsu Chicken}
\section{Sweet-Sour Chicken}
The sweetness of the sauce comes from the pineapple juice and the pineapple
chunks, while the sour taste is attributed to the rice vinegar. Balance those
to your liking. Additionally, the pineapple juice, if not on hand, may be
replaced by a sugar syrup.

The colour of the sauce ranges from an intense red to a dark brown depending on
the ingredients used. Industrial food colouring will yield a bright red. Tomato
products, such as concentrate, puree, ketchup, will result in dark red. The
addition of soy sauce will darken the colour farther into browns.

The chicken may be battered and fried, however, this recipe is intended to be
as quick and cheap as possible so that it is possible to make it when running
short on time, hence this step is omitted.

\header{Ingredients}

\ingredientsection{Chicken and Vegetables}
\begin{enumerate}
  \item one small chicken breast
  \item red bell pepper (may use more varieties for more vivid colours)
  \item red onion
  \item 100-150g pineapple
  \item 4 cloves of garlic
  \item 2cm knob of ginger
  \item spring onions for garnish
\end{enumerate}

\ingredientsection{Sauce}
\begin{enumerate}
  \item 120g pineapple juice
  \item 30g tomato concentrate, tomato puree, ketchup
  \item 30ml soy sauce
  \item 30ml rice vinegar
  \item MSG (optional)
  \item potato starch
\end{enumerate}

\header{Instructions}
Preheat your wok on highest heat at any point during the prep.

First, prepare all ingredients as the cooking process will be rather short and
fast paced. Cut the spring onions and set aside in a bowl. Cut bell pepper,
onion and pineapple into large chunks (at least 1cm), set aside. Mince garlic
and ginger, and set aside separated. Cut the chicken breast into large chunks
(at least 2cm).

In a bowl mix the pineapple juice, tomato product, soy sauce, vinegar and MSG.
In another bowl mix the starch with a little bit of water to make a slurry.
Remember to stir the slurry before adding it as the starch will settle at the
bottom.

To the hot wok add a little bit of oil and spread it around the wok to cover
its surface. Add chicken and fry on one side until browned lightly, then stir
so that the other side is not pink anymore and transfer to a large bowl. You
should not be taking too long to fry the chicken as the carryover cooking will
do its job.

Next, add a little bit of oil and fry the bell pepper and onion stirring
frequently. After around 1 minute, add the pineapple and fry stirring for 1
more minute. Transfer to the bowl with chicken.

Once again, add a little bit of oil and fry the aromatics (garlic and ginger)
for about 30 seconds. You may save about 30\% of your garlic for the end to
give the dish additional spice and garlicy taste. Pour in the sauce and boil
for 30 seconds. Pour the slurry slowly to thicken the sauce to desired
thickness remembering to stir constantly as the slurry will quickly set into
gel.

Once the sauce is thickened, transfer all the vegetables and the chicken to the
wok, stir and toss until thoroughly coated and cut the heat. If saved some of
the garlic, add it now and stir.

Serve with rice and garnish with the spring onion.

\section{Zuǒ Zōngtáng Jī (General Tso's Chicken)}
\section{Chénpí Jī (Orange Chicken)}
\section{Yaki Udon}
\section{Döner}

\section{Garnek (Hotpot)}
A lot of the flavour of the hotpot comes from the fat rendered from the meat,
hence it is best to pick sausage and pork belly that are high in fat content.

\header{Ingredients}
\begin{enumerate}
  \item 250g sausage
  \item 500g pork belly
  \item 1 carrot
  \item 1 red bell pepper
  \item 1/2 small white cabbage
  \item 400g potatoes
  \item 5 garlic
  \item ground cumin
  \item ground pepper
  \item 2 bay leaves
  \item coriander seeds
  \item paprika
  \item basil
  \item potato starch
\end{enumerate}

\header{Instructions}
Use a stainless steel pot.

\begin{enumerate}
  \item Cut sausage (preferably rangiri) and pork belly (cubes) into large
  chunks, then fry in the pot until lightly browned.
  \item Peel vegetables, then cut carrot (rangiri), pepper, cabbage and
  potatoes into large chunks. Add to the pot, then fill the pot with water
  until it covers the contents.
  \item Peel and mince garlic. Add all aromatics, herbs, spices. Leave to cook
  at least until the potatoes are tender.
  \item Mix potato starch with water to make a slurry, then add slowly while
  mixing to the pot to thicken lightly.
\end{enumerate}
