\chapter{Theoretical Introduction}

\section{Baker's Percentage}
A universal method of measuring the amount of an ingredient in baking is the
baker's percentage. It is a notation indicating the proportion of an ingredient
relative to the flour used in the recipe. Baker's percentage expresses the
ratio of the weight of an ingredient to the total weight of the flour as a
percentage:
$$
\textrm{baker's percentage} = 100\% \times \frac{ mass_{ingredient} }{ mass_{flour} }
$$
For example, a recipe calling for 65\% water will require 292.5g of water if
450g of flour are used ($\frac{450 \times 65\%}{100\%} = 450 \times 0.65
\approx 292.5$).

\section{Flour}
In Europe there are several prevailing systems for labeling flour types,
however, all correspond to a certain standardised process. A sample of flour is
incinerated in a laboratory oven at a very high temperature for a long time.
The amount of ash residue indicates the amount of whole grain that was present
in the flour. When the ash is measured in milligrams per 100g of flour, the
German flour types are obtained, such as 450 or 550. Similarly, the French
types are the same ash measured in milligrams per 10g of flour, e.g. 45 or 55
which correspond to the German types 450 and 550. The Italians took a different
approach and instead assigned the most commonly used flour types numbers 00, 0,
1 and 2. Those correspond to the German 405, 550, 800 and 1050. In general, the
higher type flours have higher protein content, but beyond 1100 the protein
content begins to drop slightly.

\section{Kneading}

\section{Melting Chocolate}

\section{Thickening Sauces}

\section{Carryover Cooking}
