\chapter{Theoretical Introduction}
Cooking and baking is the tastiest form of applied science. Is it a very
special discipline where physics, chemistry and biology meet, sprinkled with
light mathematics, to form an incredible experience for both, the creator and
the taster. Understanding the underlying fundamental concepts is essential to
success.

\section{Measuring Ingredients}
Measuring ingredients is the first step of the process. In cooking, we
frequently measure approximately, eyeball, etc. so that they are in the right
ballpark or add/season to taste. We consider recipes a set of guidelines rather
than a set of rules to be strictly followed. And in most cases such an approach
is perfectly reasonable.

Baking, on the other hand, is considerably less forgiving. An error of 5\% in
one of the measurements could be the difference between a resounding success
and a crushing failure (or several more hours in the kitchen). It is thus
important to minimise the error.

Nowadays, weighs are commonplace. A decent kitchen weigh is precise and
accurate to 1g, and capable of weighing up to 5kg. We can measure anything in
grams, from flour, through batter, to oil, without the need to worry about the
density, the temperature or the fact that we left our measuring cup at home.
Thus, the majority of the measurements in the recipes are in grams. If you do
not own a kitchen weigh yet, get one. It will make your life significantly
easier.

Volumetric or customary measurements frequently appear in older recipes.
Several hundred years ago, people did not commonly have access to a
standardised device for measuring weight, thus they used volume. However,
volumetric measurements, such as cups or spoons, are intrinsically
inconsistent. How much is a cup of flour? That depends on how large the cup in
question is and how compressed the flour is. How much is a spoon of milk?
Again, depends on the size of the spoon. Inconsistent measurements lead to
inconsistent results. And we all know that consistency is the key.
Nevertheless, it is important to be able to convert the customary units to
grams:
\begin{itemize}
\item a cup of flour is about 170g,
\item a cup of liquid is 240ml,
\item a large egg (or just egg) is between 55g and 65g,
\item a spoon (or tablespoon) is around 15ml,
\item a teaspoon is approximately 5ml,
\item a stick of butter is about 113g.
\end{itemize}
The above definitions may (and will) vary by region.

\subsection{Baker's Percentage}
A universal method of measuring the amount of an ingredient in baking is the
baker's percentage. It is a notation indicating the proportion of an ingredient
relative to the total weight of the flour used in the recipe:
\begin{equation*}
\textrm{baker's percentage} = 100\% \times \frac{ mass_{ingredient} }{ mass_{flour} }
\end{equation*}
For example, a recipe calling for 65\% water will require 292.5g of water if
450g of flour are used ($\frac{450 \times 65\%}{100\%} = 450 \times 0.65
\approx 292.5$).

\subsection{Measuring Flour} \label{sec:measuring-flour}

\subsection{Sugar and Spices}
If you are uncertain or have your doubts, the better approach is to be
conservative with how much you add. It is usually possible to season to taste
later on or incorporate an ingredient into the finished product another way
(e.g. sprinkle powdered sugar on top of a cake, make frosting slightly sweeter
to balance the overall sweetness), but it is impossible to reduce the amount of
an already combined ingredient. Flavour, which we worked hard to develop, will
be muted by overwhelming sweetness, saltiness or heat.

Additionally, preparing dishes less spicy, salty or sweet makes them more
universal as a taster can choose to season it more, to satisfy their palate, or
eat it mild. A great example is the thai cuisine which is, for some,
notoriously hot and, despite being delicious, inedible.

\section{Flour}
In Europe there are several prevailing systems for labeling flour types,
however, all correspond to a certain standardised process. A sample of flour is
incinerated in a laboratory oven at a very high temperature for a long time.
The amount of ash residue indicates the amount of whole grain that was present
in the flour.

When the ash is measured in milligrams per 100g of flour, the German flour
types are obtained, such as 450 or 550. Similarly, the French types are the
same ash measured in milligrams per 10g of flour, e.g. 45 or 55 which
correspond to the German types 450 and 550. The Italians took a different
approach and instead assigned the most commonly used flour types numbers 00, 0,
1 and 2. Those correspond to the German 405, 550, 800 and 1050. In general, the
higher type flours have higher protein content, but beyond 1100 the protein
content begins to drop slightly.

The number (or the ash content) indicates the coarsness of the flour.

\section{Gluten}
\subsection{Autolyse}
\subsection{Kneading}

\section{The Water Bath Method}
\subsection{Melting Chocolate}
\subsection{Making Custard}

\section{Thickening Sauces}

\section{Carryover Cooking}
