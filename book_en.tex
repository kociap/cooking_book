\documentclass[11pt]{report}
% ---
% title: A Short Book About Cooking and Baking
% author: Piotr Kocia
% geometry: "left=3cm,right=3cm,top=3cm,bottom=3cm"
% mainfont: Helvetica
% linestretch: 1.15
% fontsize: 11pt
% ---

% \usepackage[left=4cm, right=4cm]{geometry}

\newcommand{\header}[1]{\subsubsection*{#1}}
\newcommand{\ingredientsection}[1]{\medskip\noindent\underline{#1}\medskip}
\setcounter{tocdepth}{1}

\title{An Engineer's Book About Cooking and Baking}
\author{Piotr Kocia}

\begin{document}
\maketitle
\tableofcontents

\newpage

\chapter{Theoretical Introduction}

\section{Baker's Percentage}
A universal method of measuring the amount of an ingredient in baking is the
baker's percentage. It is a notation indicating the proportion of an ingredient
relative to the flour used in the recipe. Baker's percentage expresses the
ratio of the weight of an ingredient to the total weight of the flour as a
percentage:
$$
\textrm{baker's percentage} = 100\% \times \frac{ mass_{ingredient} }{ mass_{flour} }
$$
For example, a recipe calling for 65\% water will require 292.5g of water if
450g of flour are used ($\frac{450 \times 65\%}{100\%} = 450 \times 0.65
\approx 292.5$).

\section{Flour}
In Europe there are several prevailing systems for labeling flour types,
however, all correspond to a certain standardised process. A sample of flour is
incinerated in a laboratory oven at a very high temperature for a long time.
The amount of ash residue indicates the amount of whole grain that was present
in the flour. When the ash is measured in milligrams per 100g of flour, the
German flour types are obtained, such as 450 or 550. Similarly, the French
types are the same ash measured in milligrams per 10g of flour, e.g. 45 or 55
which correspond to the German types 450 and 550. The Italians took a different
approach and instead assigned the most commonly used flour types numbers 00, 0,
1 and 2. Those correspond to the German 405, 550, 800 and 1050. In general, the
higher type flours have higher protein content, but beyond 1100 the protein
content begins to drop slightly.

\section{Kneading}

\section{Melting Chocolate}

\section{Thickening Sauces}

\section{Carryover Cooking}

\newpage

\chapter{Pasta}
\header{Ingredients}
\header{Instructions}

\chapter{Noodles}
\section{Udon}
\header{Ingredients}
\begin{enumerate}
  \item wheat flour
  \item water 40\%
  \item salt 3-4\%
\end{enumerate}

For a single portion, 100g of flour is sufficient.

\header{Instructions}
Mix and knead the ingredients into a rough dough. Cover and let rest for at
least 30 minutes.

Once ready to make the noodles, divide into 150g balls. Using a rolling pin,
roll a ball out to a thick sheet. Then, using a pasta machine, roll it to 3mm
thickness (number 2 on Atlas 150) gradually decreasing the thickness setting on
the machine. Fold in half, roll to 3mm thickness. Fold in half from the
opposite end and roll to 3mm thickness. At this point the dough should be
roughly in a rectangular shape. If it is not, continue folding and rolling. You
may fold from both ends at the same time overlapping the folds or turn the
dough 90 degrees and continue from there. Although udon is standardised to be
at least 1.7mm in diameter (when round) or width (when square), there is no one
true udon thickness, hence roll it to your preferences.

Once done, dust generously with flour. If you have a bigoli (3.5mm round),
trenette (3.5mm square), spaghetti (2mm round) or spaghetti chitarra (2mm
square) attachment, run your sheet through. Otherwise, place your sheet on a
cutting board and fold in half twice ensuring that the layers are generously
dusted with flour. Using a sharp knife cut using little pressure and moving
your knife back and forth. Do not press as that might cause the noodle to stick
to itself. Move the cut noodle to the side enrusing it does not stick to the
other noodles (the freshly cut sides are sticky).

Boil water, at least 1L as the boiling process is long. Once the water is
boiling, drop your noodles in, stir after 1 minute and then occasionally every
few minutes. Boiling udon takes at least 10 minutes, but it heavily depends on
the thickness. Thicker noodles require longer boiling.

Strain your noodles and cool them with cold water or in an ice bath.

\section{Soba}

\chapter{Breads, Buns and Rolls}
\section{Pre-Ferment Bread}

\header{Ingredients}
For a single loaf

\ingredientsection{Pre-Ferment (Poolish)}
\begin{enumerate}
  \item 150g flour (12\%+ protein or type 650+)
  \item 150g water (room temperature)
  \item 1g fresh yeast
\end{enumerate}

\ingredientsection{Dough}
\begin{enumerate}
  \item 400g flour (12\%+ protein or type 650+)
  \item 280g water (35C)
  \item 6g fresh yeast
  \item 10g salt
\end{enumerate}

\header{Instructions}
Prepare the poolish by mixing all ingredients in a glass jar or a glass
container. Leave to ferment in room temperature for 16-24 hours.

Once the poolish is ready, mix it with the ingredients for the dough in a large
bowl until homogenous. Leave to rest for around 30 minutes, then fold several
times onto itself. Repeat at least one more time and leave to rest for 30
minutes.

Take the dough out of the bowl onto a surface lightly sprinkled with flour.
Fold the dough onto itself several times by grabbing an edge of the dough,
stretching it up and bringing it to the opposite side. Work clockwise or
counterclockwise. Flip the dough upside-down and shape into a ball to form a
smooth surface by pulling the dough under itself. Keep the seam at the bottom
of the ball throughout the process.

Prepare a proofing basket or line a large bowl with a clean kitchen towel and
toss very generously with flour (otherwise the dough will stick to the towel).
Toss the top of the dough with flour and place it seam side up in the proofing
basket. Proof at room temperature for 30 minutes to one hour.

While the dough is proofing, heat the oven and a dutch oven (you may use a
large steel pot with a lid as a replacement) to 250C. Once hot, sprinkle the
bottom of the dutch oven with flour and carefully place the dough in it. Do not
drop the dough as it will lose the gases built up during proofing resulting in
a flatter loaf. Score the top of the bread deeply. Bake covered for 20 minutes,
remove the cover and bake for 20 or more minutes to achieve dark brown crust.

\section{Sourdough Bread}
\section{Pide}
\section{Burger Bun}
This recipe makes 6 buns (150g).

\header{Ingredients}
\begin{enumerate}
  \item 125g milk (lukewarm or room temperature)
  \item 125g water (lukewarm or room temperature)
  \item 20g yeast
  \item 15g sugar
  \item 1 whole egg + 1 egg yolk
  \item 10g salt
  \item 525g flour
  \item 65g butter (softened)
  \item sesame (optional)
  \item 1 whole egg for egg wash
\end{enumerate}

\header{instructions}
\begin{enumerate}
  \item Mix together milk, water, yeast, sugar, salt and eggs.
  \item Add flour, mix, then knead for up to 5 minutes.
  \item Knead the butter into the dough and continue kneading until the dough
  does not stick anymore.
  \item Form into a ball and leave for at least an hour to rise.
  \item Divide into 6 even portions and shape into balls. Place on sheets of
  parchment paper, then lightly coat the tops of the buns with oil and gently
  flatten with the bottom of a glass.
  \item Cover the buns with cloth and leave to proof for 30 minutes to an hour.
  \item Preheat the oven to 190C.
  \item Mix together an egg and a little water (about a tablespoon), then brush
  the tops of the buns with the egg wash. Coat generously with sesame.
  \item Bake for 17-20 minutes.
\end{enumerate}

\section{Hoagie Roll}

\chapter{Pizza}
\section{Pizza Napoletana}

\header{Ingredients}
\begin{enumerate}
  \item wheat flour (11-12\% protein or type 450)
  \item 60\%+ water (room temperature)
  \item 0.5\% fresh yeast
  \item 2\% salt
\end{enumerate}

\header{Measuring}
It is important to measure the appropriate amounts of ingredients, so that we
may divide the dough evenly without any leftover (too large pieces will result
in pizzas with a thick bottom, while too small pieces will result in smaller
pizzas). We will be dividing the dough into $280 \pm 5$g pieces. For example,
if we want to make 4 pizzas, we should make 1120g of dough. The mass of the
  entire dough is approximately the sum of the masses of the ingredients, hence
  we may calculate the amount of flour needed:
$$
mass_{flour} = mass_{dough} \times \frac{100\%}{ percent_{flour} + percent_{water} + percent_{salt} + percent_{yeast} }
$$
For a 60\% hydration dough we need $\frac{1120}{1 + 0.6 + 0.02 + 0.005} =
\frac{1120}{1.625} = 689.3$g flour.

\header{Instructions}
We mix the ingredients in a large bowl, then knead using any technique for
about 20 minutes. For high hydration dough we may instead choose the folding
technique and perform it periodically during the resting period. We leave the
dough for at least 2h to rest, but due to it being a low yeast content dough,
we may leave it for 5h or more.

After the dough rests, we prepare containers to store it. We may use small
containers for each piece of dough or large fermentation boxes (as seen in
commercial pizzerias) for multiple. Cover the containers with a thin layer of
oil. Divide the dough into $280 \pm 5$g pieces, shape into balls with a smooth
surface (extremely important) by, for instance, folding the dough into itself
and place them in the containers. If storing multiple ball in one container,
leave at least 5cm of space inbetween each pair. Store in the fridge and let
ferment for at least 8h. It is possible to store the dough in the fridge for up
to a week.

After taking out of the fridge, let rest for around 30 minutes or until it
reaches around 16C. In the meantime preheat the oven to the highest available
temperature. Carefully take one dough ball out of a container (it is important
to not push the gases out of the dough at this stage) onto flour and stretch
using any technique while being cautious to not press the gases out of the
crust. Bake until the crust turns crispy (for reference, 6-8 minutes in 250C).

\chapter{Burgers}
\section{Beef Burgers}
\section{Chicken Burgers}
\section{Vegetarian Burgers}

\chapter{Sandwiches}
\section{Chopped Cheese}
\section{Chicken Sandwich}

\chapter{Dumplings}
\section{Pierogi}
\section{Gyoza}

\chapter{Sides}
\section{Spring Rolls}
\subsection{Wrappers}

\header{Ingredients}
\begin{enumerate}
  \item wheat flour
  \item 50\% water
  \item 10\% oil
  \item 2\% salt
\end{enumerate}

\header{Instructions}
\begin{enumerate}
  \item Mix the ingredients into rough dough and leave to autolyse for 30
  minutes to an hour.
  \item Knead the dough lightly to ensure it is homogenous, form into a ball
  and leave to rest for 30 minutes up to overnight in the fridge.
  \item Measure out 35g portions of the dough and roll them into 1mm thick
  sheets, then cut into squares. Dust generously with flour to ensure they do
  not stick when stacked. If using a pasta machine, roll the entire dough into
  1mm thick sheet, then cut into squares.
\end{enumerate}

\subsection{Filling}
\header{Ingredients}
\begin{enumerate}
  \item 250g ground pork
  \item 1 carrot
  \item 2 leaves of napa cabbage
  \item 1/2 red bell pepper
  \item scallion
  \item 2 cloves of garlic
\end{enumerate}

\header{Marinade}
\begin{enumerate}
  \item 15ml soy sauce
  \item 5ml sesame oil
  \item 5g sugar
\end{enumerate}

\header{Sauce}
\begin{enumerate}
  \item 30ml fish sauce
  \item 30ml oyster sauce
  \item 15g sugar
\end{enumerate}

\header{Instructions}
\begin{enumerate}
  \item Mix the marinade ingredients together, then mix the marinade with the
  ground pork.
  \item Mix the sauce ingerdients to allow the sugar to dissolve.
  \item Cut all vegetables into thin slices.
  \item Heat up your wok, then fry the meat on low heat until lightly browned.
  \item
\end{enumerate}

\chapter{Main Courses}
\section{Chicken Broccoli}
\section{Katsu Chicken}
\section{Sweet-Sour Chicken}
The sweetness of the sauce comes from the pineapple juice and the pineapple
chunks, while the sour taste is attributed to the rice vinegar. Balance those
to your liking. Additionally, the pineapple juice, if not on hand, may be
replaced by a sugar syrup.

The colour of the sauce ranges from an intense red to a dark brown depending on
the ingredients used. Industrial food colouring will yield a bright red. Tomato
products, such as concentrate, puree, ketchup, will result in dark red. The
addition of soy sauce will darken the colour farther into browns.

The chicken may be battered and fried, however, this recipe is intended to be
as quick and cheap as possible so that it is possible to make it when running
short on time, hence this step is omitted.

\header{Ingredients}

\ingredientsection{Chicken and Vegetables}
\begin{enumerate}
  \item one small chicken breast
  \item red bell pepper (may use more varieties for more vivid colours)
  \item red onion
  \item 100-150g pineapple
  \item 4 cloves of garlic
  \item 2cm knob of ginger
  \item spring onions for garnish
\end{enumerate}

\ingredientsection{Sauce}
\begin{enumerate}
  \item 120g pineapple juice
  \item 30g tomato concentrate, tomato puree, ketchup
  \item 30ml soy sauce
  \item 30ml rice vinegar
  \item MSG (optional)
  \item potato starch
\end{enumerate}

\header{Instructions}
Preheat your wok on highest heat at any point during the prep.

First, prepare all ingredients as the cooking process will be rather short and
fast paced. Cut the spring onions and set aside in a bowl. Cut bell pepper,
onion and pineapple into large chunks (at least 1cm), set aside. Mince garlic
and ginger, and set aside separated. Cut the chicken breast into large chunks
(at least 2cm).

In a bowl mix the pineapple juice, tomato product, soy sauce, vinegar and MSG.
In another bowl mix the starch with a little bit of water to make a slurry.
Remember to stir the slurry before adding it as the starch will settle at the
bottom.

To the hot wok add a little bit of oil and spread it around the wok to cover
its surface. Add chicken and fry on one side until browned lightly, then stir
so that the other side is not pink anymore and transfer to a large bowl. You
should not be taking too long to fry the chicken as the carryover cooking will
do its job.

Next, add a little bit of oil and fry the bell pepper and onion stirring
frequently. After around 1 minute, add the pineapple and fry stirring for 1
more minute. Transfer to the bowl with chicken.

Once again, add a little bit of oil and fry the aromatics (garlic and ginger)
for about 30 seconds. You may save about 30\% of your garlic for the end to
give the dish additional spice and garlicy taste. Pour in the sauce and boil
for 30 seconds. Pour the slurry slowly to thicken the sauce to desired
thickness remembering to stir constantly as the slurry will quickly set into
gel.

Once the sauce is thickened, transfer all the vegetables and the chicken to the
wok, stir and toss until thoroughly coated and cut the heat. If saved some of
the garlic, add it now and stir.

Serve with rice and garnish with the spring onion.

\section{Zuǒ Zōngtáng Jī (General Tso's Chicken)}
\section{Chénpí Jī (Orange Chicken)}
\section{Yaki Udon}
\section{Döner}

\section{Hotpot (Garnek)}
A lot of the flavour of the hotpot comes from the fat rendered from the meat,
hence it is best to pick sausage and pork belly that are high in fat content.

\header{Ingredients}
\begin{enumerate}
  \item 250g sausage
  \item 500g pork belly
  \item 1 carrot
  \item 1 red bell pepper
  \item 1/2 small white cabbage
  \item 400g potatoes
  \item 5 garlic
  \item ground cumin
  \item ground pepper
  \item 2 bay leaves
  \item coriander seeds
  \item paprika
  \item basil
  \item potato starch
\end{enumerate}

\header{Instructions}
Use a stainless steel pot.

\begin{enumerate}
  \item Cut sausage (preferably rangiri) and pork belly (cubes) into large
  chunks, then fry in the pot until lightly browned.
  \item Peel vegetables, then cut carrot (rangiri), pepper, cabbage and
  potatoes into large chunks. Add to the pot, then fill the pot with water
  until it covers the contents.
  \item Peel and mince garlic. Add all aromatics, herbs, spices. Leave to cook
  at least until the potatoes are tender.
  \item Mix potato starch with water to make a slurry, then add slowly while
  mixing to the pot to thicken lightly.
\end{enumerate}

\chapter{Cookies and Biscuits}

\section{Savoiardi \label{Savoiardi}}


\section{Oatmeal Cookies}
\header{Ingredients}
For about 18 cookies:
\begin{enumerate}
  \item 250g oatmeal
  \item 200g flour (type 450-650)
  \item 150g butter
  \item 2 eggs
  \item 4g salt
  \item 4g cinnamon
  \item dried fruits (for example cranberry)
  \item 100g 64\% chocolate or chocolate chips
  \item brown sugar, honey or caramel (optional)
  \item water (optional)
\end{enumerate}

\header{Instructions}
Preheat your oven to about 165C.

Take a steel pot and set it over medium heat. Add butter and brown until butter
solids appear and the liquid turns golden brown. It is important to transfer
the liquid butter to a bowl or another container, or cool the pot in a water
bath until it is around room temperature because otherwise the butter might
burn from the heat stored in the pot.

Blend half the oatmeal coarsely, transfer to a container, then blend the dried
fruits until they turn into tiny pieces. We do not want to turn them into a
paste, though. Cut the chocolate int rough 5mm squares.

Take a large bowl and mix eggs and sugar. Add butter and mix until combined,
then add flour and whisk until the mass turns smooth. Add your remaining
ingredients and mix until the mass is uniform. If the mass is crumbly, add some
water.

Line a baking tray with parchment paper. Scoop a small portion of the dough,
form into a rough ball, then lay on the tray and press down until about 1cm
thick. Repeat to make 9 cookies. Bake for 17 minutes or until the bottom of the
cookies is brown.

\chapter{Desserts}

\section{Tiramisu}

\header{Ingredients}
\begin{enumerate}
  \item 4 eggs
  \item 50g sugar
  \item 500g mascarpone
  \item ~14 savoiardi
  \item 300ml espresso (lukewarm or cold)
  \item cocoa powder
\end{enumerate}

\header{Instructions}
If you follow the order of whipping, you won't have to clean your mixer.
Otherwise, make sure to not stain the whites with any yolks or mascarpone,
and to not introduce mascarpone to your yolks.
\begin{enumerate}
  \item Separate whites from yolks.
  \item Whip the whites until stiff.
  \item Add sugar to yolks and whip until pale. They will significantly
  increase in volume, so be mindful of the container you use.
  \item Whip the mascarpone until soft.
  \item Fold the yolks into the whites, then fold mascarpone into the mix until
   smooth. Do not overmix as the eggs will lose the air.
  \item Lay the first layer of savoiardi dipping them in the coffee (do not soak
   them!) just before placing them.
  \item Place a thick layer of the mascarpone.
  \item Repeat with the next layer.
  \item Refrigerate for several hours, preferably overnight.
  \item Dust with cocoa powder before cutting.
\end{enumerate}

\section{Panna Cotta}
Panna Cotta may be served and garnished in a variety of ways, e.g. layered in a
glass or inverted onto a plate with a side of fruits. In this recipe we will be
making a 3-layer Panna Cotta in whiskey glasses, however, the recipe for the
cream itself may be used to make any variety of Panna Cotta.
\header{Ingredients}
For 3 servings:

\ingredientsection{Cream}
\begin{enumerate}
  \item 500ml cream 36\% or 30\%
  \item 25g sugar
  \item 7g gelatin (powder, for other kinds use the appropriate method to
  dissolve)
  \item vanilla bean (optional)
\end{enumerate}

\ingredientsection{Chocolate}
\begin{enumerate}
  \item 30g dark chocolate 64\%+
  \item 20g butter
  \item confectioner sugar (optional)
\end{enumerate}

\ingredientsection{Fruit Sauce}
\begin{enumerate}
  \item 100g strawberries, peaches or any other fresh fruits
  \item sugar (optional)
\end{enumerate}

\ingredientsection{Garnish}
\begin{enumerate}
  \item 1 large strawberry, 1 slice of peach or a part of the fruit used to
  make the sauce
  \item 2 mint leaves
\end{enumerate}

\header{Instructions}
In a small bowl mix 3 spoons of cream with the gelatin. To a large pot, add the
sugar and the rest of the cream and set over small heat. Use a thermometer to
measure the temperature and stir constantly. Pour in the cream mixed with the
gelatin and mix until thoroughly incorporated. If using vanilla or other
spices, add them now. Once the temperature reaches slightly below 80C, cut the
heat and distribute evenly into 400ml whiskey glasses. Cover tightly with
plastic foil and place in the fridge for at least 4h.

Once the cream sets and you are ready to serve the dessert, prepare the
chocolate and the fuit sauce. Melt the chocolate and the butter in a water
bath. Blend the fruits. You may optionally cook the fruits before blending.
Prepare the garnish. Remove the cover from a glass and wipe any moisture from
the inside. Layer chocolate thinly as it has quite an intense flavour. Pour the
sauce gently on top and garnish with fruit and mint.

\section{Brownies}
There are two types of brownies - fudgy and cakey. Fudgy brownies are chewy,
gooey, moist, while cakey brownies resemble a very dense genoise. It is up to
your personal preference which type you will make.

The proportions of fats (from butter and chocolate) and flour will vary the
fudgyness of the brownies. For fudgy brownies add more fat (butter and
chocolate) and for cakey brownies add more flour. The mixing technique, the
baking time and the temperature also affect the texture of the brownies. For
fudgy brownies, barely mix the ingredients and bake shorter, while for cakey
brownies do the exact opposite - ensure the ingredients are thoroughly
incorporated, preferably using a mixer, and the brownies are baked for a longer
time. Additionally, for the cakey brownies, add baking powder to improve the
height and the fluffiness .

Farthermore, using less sugar will prevent the sweetness from overwhelming the
bitter taste of the chocolate and the cocoa resulting in a richer flavour
variety (the following recipe already reflects that). However, the sugar, when
baking, carmelises at the top forming a light crust. Adding too little sugar
will result in no crust and extremely bitter brownies.

The following recipe is for a 22x22cm baking pan of fudgy brownies.

\subsection*{Ingredients}
\begin{enumerate}
  \item 180g 64\% chocolate
  \item 115g butter
  \item 110g white sugar
  \item 3 eggs
  \item 60g flour
  \item 50g cocoa powder
  \item 7g salt
  \item 120g 64\% chopped chocolate or chocolate chips
\end{enumerate}

Preheat your oven to 165C and line a 22x22cm baking pan with parchment paper.

In a water bath, melt the chocolate and the butter stirring occasionally to
combine. Once fully melted, take off the heat source, add sugar and salt, and
mix. Beat the living hell out of the eggs with a mixer (at least 10 minutes).
Pour in the chocolate mixing continously on low speed. Add the dry ingredients
(flour, cocoa powder) and the chopped chocolate or chocolate chips. Fold gently
until no flour pockets remain (we do not want to deflate the batter).

Pour the batter into the pan and bake for 18-20 minutes. Let cool in the pan on
a wire rack for several hours before cutting. Cut into 9 pieces.

\section{Crescent Rolls}

\chapter{Pies}
\section{Apple Pie}
\begin{enumerate}
  \item 400g flour
  \item 150g butter
  \item 2 egg yolks (about 30g)
  \item 150g cream 30\% or 36\%
  \item 8g salt
  \item 75g sugar
\end{enumerate}

\chapter{Cakes}
\section{Genoise}
\section{Shortcake}

\section{Carrot Cake}
This recipe is scaled to a 15cm (diameter) cake tin.

\subsection*{Ingredients}
\begin{enumerate}
  \item 120g eggs (room temperature)
  \item 70g sugar (prefer brown for better moisture retention, but white is also
  perfect)
  \item 77g oil (odourless, e.g. canola)
  \item 110g flour
  \item 55g almond powder (substitute with flour if not available)
  \item 3g cinnamon
  \item 6.5g baking powder
  \item 2g salt
  \item 170g carrots (mass after peeling)
  \item 50g walnuts
  \item 50g raisins
\end{enumerate}

\subsection*{Instructions}
\begin{enumerate}
  \item Line the bottom and the sides of your tin with parchment paper.
  \item Preheat oven to 165C.
  \item Submerge raisins in rum/water/liquid of your choice and leave to soak.
  \item Grate carrots on the fine side of your grater.
  \item Chop walnuts.
  \item Whisk eggs and sugar.
  \item Add oil and whisk until well combined.
  \item Add dry ingredients (flour, almond powder, cinnamon, baking powder,
  salt) and mix well with a spatula.
  \item Add carrots, walnuts, drained raisins and mix.
  \item Pour into the tin. Drop the tin from about 10cm to remove large air
  bubbles in the batter.
  \item Bake 40-45 mins.
  \item Invert onto a cooling rack, remove the tin and the parchment paper.
  cool for 10 mins, then flip up-side-down and continue cooling until edible or
  room temperature.
  \item Slice into 3 layers of 1.5cm to 2cm height. Layer with cream cheese
  frosting.
\end{enumerate}

\section{Cream Cheese Frosting}

\header{Ingredients}
\begin{enumerate}
  \item 320g cream cheese (room temperature)
  \item 120g powder sugar
  \item 160g butter (room temperature)
  \item 3.5g lemon zest
  \item 30g heavy cream
\end{enumerate}

\header{Instructions}
\begin{enumerate}
 \item Mix the cream cheese with the powder sugar.
 \item In another bowl mix the the butter to soften it. This helps ensure there
 will be no clumps of butter in the frosting.
 \item Add the butter to the cream cheese and mix until homogenous.
 \item Add lemon zest and heavy cream. Mix until homogenous.
 \item Chill in the fridge for about 20 minutes before using.
\end{enumerate}

\end{document}
