\chapter{Pastry}
\section{Tart Shell (Pâte Sablée)} \label{pate-sablee} \label{tart-shell}
Pâte Sablée (French shortcrust pastry) is one of the three basic kinds of
"crumbly" pastry (pâte friable). The crunchy and crumbly texture combined with
a slight sweet tone makes this dough ideal for sweet tarts.

\header{Ingredients}
\begin{itemize}
  \item 100g butter (room temperature)
  \item 72g powder sugar
  \item 20g almond powder (substitute flour if no almond powder)
  \item 30g corn/potato starch
  \item 36g whole eggs (whisked, room temperature)
  \item 170g flour (400 or 600)
  \item 2g salt
\end{itemize}

\header{Instructions}
Use a mixer. Throughout the entire process, all mixing should be done "lightly",
that is shortly and on low speed. We do not want to incorporate much air into
the dough so as to prevent it from rising in the oven.
\begin{enumerate}
  \item Beat the butter.
  \item Sift in the powdered sugar, the almond powder and the starch. Mix until
  combined.
  \item Mix the eggs in, adding in batches to prevent lumping of the dough.
  \item Sift in the flour and mix until rough crumbly dough forms.
  \item Transfer the dough onto a silicon mat.
  \item Work the dough using a bench scraper or any other tool until it is
  homogenous. It is also possible to do this by hand, however, the warmth will
  melt the butter, altering the texture.
  \item Place the dough between two sheets of parchment paper. This eliminates
  the need to use additional flour and makes the scraps from cutting reusable.
  \item Roll out the dough to 2mm thickness.
  \item Chill the dough in a fridge for 1-2 hours. This step is optional, but
  is highly recommended as soft dough is difficult to work with.
  \saveenum
\end{enumerate}
It is best to prepare the tart shells on a perforated silicone mat on which we
will also bake. Moving them onto the mat from another surface might prove
difficult. Make sure your mat is heat resistant up to \SI{180}{\celsius}.
\begin{enumerate}
  \contenum
  \item Cut the bottoms of the shells out (press the tart rings into the dough).
  \item Cut strips slightly taller than the tart ring. Tightly line the sides
  of the tart rings with the strips. Make sure the dough is not too cold as it
  might snap. Cut the excess leaving a slight overlap, then press the ends of
  the strips together.
  \item Trim the excess dough from the top, cutting towards the outside.
  \item Bake at \SI{160}{\celsius} for \textasciitilde 15 minutes.
\end{enumerate}
The tart shells may be stored for several weeks in a dry environment.

\section{Crème Pâtissière (Pastry Cream)}
Crème Pâtissière is a type of thick custard made of milk and eggs. This recipe
incorporates sugar, making the cream perfect to be used directly or as a base
of another cream in tarts, èlcairs, choux, etc.

\header{Ingredients}
\begin{itemize}
  \item 265g milk
  \item vanilla bean
  \item 50g sugar
  \item 65g egg yolks
  \item 20g corn/potato starch
  \item 20g butter
\end{itemize}

\header{Instructions}
Use a whisk, preferably a narrow one.
\begin{enumerate}
  \item Pour milk into a saucepan, add the vanilla seeds and the bean, then heat
  to around \SI{60}{\celsius} to infuse the vanilla flavour.
  \item Whisk the yolks, sugar and the starch until smooth.
  \item Once infused, pour the milk into the egg mixture in small batches while
  whisking.
  \item Strain the whole mixture into the saucepan.
  \item Set it over a small heat and whisk continously. The cream whill thicken
  considerably around the boiling point. Once it does, whisk vigorously for
  about a minute.
  \item Take off the heat and whisk for another minute. The cream should loosen
  up and have a shiny appearance.
  \item Add the butter and blend until the cream becomes smooth.
  \item Transfer to another container (optional) and cover with the surface with
  plastic foil to prevent forming a skin.
  \item Refrigerate. Once cold, the cream will have a jelly-like consistency.
  Whisk before using.
\end{enumerate}

\section{Tarte au Citron (Lemon Tart)}
This recipe makes 4 tarts, thus 4 $\diameter$10cm \fullref{tart-shell} are required.

\header{Ingredients (Lemon Curd)}
\begin{itemize}
  \item 125g lemon juice
  \item zest of the lemons (optional)
  \item 65g egg yolks
  \item 110g whole eggs
  \item 95g sugar
  \item 2.5g gelatin
  \item 80g butter (optional)
\end{itemize}
Lemon zest has a rich lemon flavour and since we already are using the lemons,
we might also zest them before squeezing the juice. While optional, the zest
improves the flavour and is recommended. Butter improves the mouthfeel - makes
the whole curd smoother.

\header{Instructions}
Vigorous stirring, especially using metal utensils, in a metal pot will cause a
metallic aftertaste in the curd due to the acid from the lemon reacting with
the metal. Prefer using a silicone spatula instead of a metal whisk.
\begin{enumerate}
  \item Zest the lemons.
  \item Squeeze the lemons.
  \item Mix the gelatin with 20g of the lemon juice.
  \item Combine the remaining lemon juice, zest, yolks, eggs and sugar in a pot,
    then set it over low heat. Stir gently and constantly to prevent curdling.
  \item Once the curd thickens considerably, remove from the heat and continue
    stirring for up to a minute.
  \item While still hot (around \SI{50}{\celsius}), add the gelatin and mix to
    dissolve.
  \item Add the butter and blend with an immersion blender until smooth.
  \item Fill the shells with the curd and move to a fridge to chill and set for
    about an hour.
\end{enumerate}

\header{Ingredients (Lemon Meringue)}
\begin{itemize}
  \item 30g water \tikzmark{sugar-syrup-top}
  \item 70g sugar \tikzmark{sugar-syrup-bot}
  \item 25g sugar \tikzmark{french-meringue-top}
  \item 65g egg whites \tikzmark{french-meringue-bot}
  \item 15g lemon juice
  \item 1.5g gelatin
  \item lemon zest
\end{itemize}

\begin{tikzpicture}[overlay, remember picture]
  \node (sugar-syrup-top) at (pic cs:sugar-syrup-top) {\vphantom{h}};
  \node (sugar-syrup-bot) at (pic cs:sugar-syrup-bot) {\vphantom{g}};
  \draw [decoration = {brace, amplitude = 0.6em}, decorate, very thick, black]
    let
      \p1 = (sugar-syrup-top.north),
      \p2 = (sugar-syrup-bot.south)
    in
      ({max(\x1, \x2)}, \y1) -- ({max(\x1, \x2)}, \y2)
      node [midway, right, xshift = 0.7em] {Sugar Syrup};

  \node (french-meringue-top) at (pic cs:french-meringue-top) {\vphantom{h}};
  \node (french-meringue-bot) at (pic cs:french-meringue-bot) {\vphantom{g}};
  \draw [decoration = {brace, amplitude = 0.6em}, decorate, very thick, black]
    let
      \p1 = (french-meringue-top.north),
      \p2 = (french-meringue-bot.south)
    in
      ({max(\x1, \x2)}, \y1) -- ({max(\x1, \x2)}, \y2)
      node [midway, right, xshift = 0.7em] {French Meringue};
\end{tikzpicture}

\header{Instructions}
\begin{enumerate}
  \item Dissolve the gelatin in the lemon juice by heating it up to
    \SI{40}{\celsius}.
  \item Make the sugar syrup by comining the water and the sugar in a pot, then
    setting it on low heat until it reaches around \SI{120}{\celsius}.
  \item In the meantime, make the french meringue by beating the egg whites
    with the sugar.
  \item Pour the hot sugar syrup into the meringue in small batches while
    mixing constantly.
  \item Immediately after add the lemon juice with the gelatine and mix to
    incorporate.
  \item Once cooled, transfer to a pastry bag with a tip of your choice.
  \item Pipe on top of the curd.
  \item Sprinkle with the lemon zest.
  \item Move to a fridge to chill and set for about an hour.
\end{enumerate}

\section{Crescent Rolls}
\header{Ingredients}

\section{Pączki (Berliner, Doughnuts)}
\header{Ingredients}
