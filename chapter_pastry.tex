\chapter{Pastry}
\section{Tart Shell (Pâte Sablée)} \label{pate-sablee} \label{tart-shell}
Pâte Sablée (French shortcrust pastry) is one of the three basic kinds of
"crumbly" pastry (pâte friable). The crunchy and crumbly texture combined with
a slight sweet tone makes this dough ideal for sweet tarts.

\header{Ingredients}
\begin{enumerate}
  \item 100g butter (room temperature)
  \item 72g powder sugar
  \item 20g almond powder (substitute flour if no almond powder)
  \item 30g corn/potato starch
  \item 36g whole eggs (whisked, room temperature)
  \item 170g flour (400 or 600)
  \item 2g salt
\end{enumerate}

\header{Instructions}
Use a mixer. Throughout the entire process, all mixing should be done "lightly",
that is shortly and on low speed. We do not want to incorporate much air into
the dough so as to prevent it from rising in the oven.
\begin{enumerate}
  \item Beat the butter.
  \item Sift in the powdered sugar, the almond powder and the starch. Mix until
  combined.
  \item Mix the eggs in, adding in batches to prevent lumping of the dough.
  \item Sift in the flour and mix until rough crumbly dough forms.
  \item Transfer the dough onto a silicon mat.
  \item Work the dough using a bench scraper or any other tool until it is
  homogenous. It is also possible to do this by hand, however, the warmth will
  melt the butter, altering the texture.
  \item Place the dough between two sheets of parchment paper. This eliminates
  the need to use additional flour and makes the scraps from cutting reusable.
  \item Roll out the dough to 2mm thickness.
  \item Chill the dough in a fridge for 1-2 hours. This step is optional, but
  is highly recommended as soft dough is difficult to work with.
  \saveenum
\end{enumerate}
It is best to prepare the tart shells on a perforated silicone mat on which we
will also bake. Moving them onto the mat from another surface might prove
difficult. Make sure your mat is heat resistant up to \SI{180}{\celsius}.
\begin{enumerate}
  \contenum
  \item Cut the bottoms of the shells out (press the tart rings into the dough).
  \item Cut strips slightly taller than the tart ring. Tightly line the sides
  of the tart rings with the strips. Make sure the dough is not too cold as it
  might snap. Cut the excess leaving a slight overlap, then press the ends of
  the strips together.
  \item Trim the excess dough from the top, cutting towards the outside.
  \item Bake at \SI{160}{\celsius} for \textasciitilde 15 minutes.
\end{enumerate}
The tart shells may be stored for several weeks in a dry environment.

\section{Tarte au Citron (Lemon Tart)}

\header{Ingredients (Lemon Custard)}
\header{Ingredients (Lemon Meringue)}

\section{Crescent Rolls}
\header{Ingredients}

\section{Pączki (Berliner, Doughnuts)}
\header{Ingredients}
