\chapter{Desserts}

\section{Tiramisu}

\header{Ingredients}
\begin{enumerate}
  \item 4 eggs
  \item 50g sugar
  \item 500g mascarpone
  \item ~14 savoiardi (for a recipe, see section \ref{Savoiardi})
  \item 300ml espresso (lukewarm or cold)
  \item cocoa powder
\end{enumerate}

\header{Instructions}
The coffee must not be hot as otherwise the savoiardi dipped in it will melt the
mascarpone creme.

If you follow the order of whipping, you won't have to clean your mixer.
Otherwise, make sure to not stain the whites with any yolks or mascarpone,
and to not introduce mascarpone to your yolks.
\begin{enumerate}
  \item Separate whites from yolks.
  \item Whip the whites until stiff.
  \item Add sugar to yolks and whip until pale. They will significantly
  increase in volume, so be mindful of the container you use.
  \item Whip the mascarpone until soft.
  \item Fold the yolks into the whites, then fold mascarpone into the mix until
   smooth. Do not overmix as the eggs will lose the air.
  \item Lay the first layer of savoiardi dipping them in the coffee (do not soak
   them!) just before placing them.
  \item Place a thick layer of the mascarpone.
  \item Repeat with the next layer.
  \item Refrigerate for several hours, preferably overnight.
  \item Dust with cocoa powder before serving.
\end{enumerate}

\section{Panna Cotta}
Panna Cotta may be served and garnished in a variety of ways, e.g. layered in a
glass or inverted onto a plate with a side of fruits. In this recipe we will be
making a 3-layer Panna Cotta in whiskey glasses, however, the recipe for the
cream itself may be used to make any variety of Panna Cotta.
\header{Ingredients}
For 3 servings:

\ingredientsection{Cream}
\begin{enumerate}
  \item 500ml cream 36\% or 30\%
  \item 25g sugar
  \item 7g gelatin (powder, for other kinds use the appropriate method to
  dissolve)
  \item vanilla bean (optional)
\end{enumerate}

\ingredientsection{Chocolate}
\begin{enumerate}
  \item 30g dark chocolate 64\%+
  \item 20g butter
  \item confectioner sugar (optional)
\end{enumerate}

\ingredientsection{Fruit Sauce}
\begin{enumerate}
  \item 100g strawberries, peaches or any other fresh fruits
  \item sugar (optional)
\end{enumerate}

\ingredientsection{Garnish}
\begin{enumerate}
  \item 1 large strawberry, 1 slice of peach or a part of the fruit used to
  make the sauce
  \item 2 mint leaves
\end{enumerate}

\header{Instructions}
In a small bowl mix 3 spoons of cream with the gelatin. To a large pot, add the
sugar and the rest of the cream and set over small heat. Use a thermometer to
measure the temperature and stir constantly. Pour in the cream mixed with the
gelatin and mix until thoroughly incorporated. If using vanilla or other
spices, add them now. Once the temperature reaches slightly below 80C, cut the
heat and distribute evenly into 400ml whiskey glasses. Cover tightly with
plastic foil and place in the fridge for at least 4h.

Once the cream sets and you are ready to serve the dessert, prepare the
chocolate and the fuit sauce. Melt the chocolate and the butter in a water
bath. Blend the fruits. You may optionally cook the fruits before blending.
Prepare the garnish. Remove the cover from a glass and wipe any moisture from
the inside. Layer chocolate thinly as it has quite an intense flavour. Pour the
sauce gently on top and garnish with fruit and mint.

\section{Brownies}
There are two types of brownies - fudgy and cakey. Fudgy brownies are chewy,
gooey, moist, while cakey brownies resemble a very dense genoise. It is up to
your personal preference which type you will make.

The proportions of fats (from butter and chocolate) and flour will vary the
fudgyness of the brownies. For fudgy brownies add more fat (butter and
chocolate) and for cakey brownies add more flour. The mixing technique, the
baking time and the temperature also affect the texture of the brownies. For
fudgy brownies, barely mix the ingredients and bake shorter, while for cakey
brownies do the exact opposite - ensure the ingredients are thoroughly
incorporated, preferably using a mixer, and the brownies are baked for a longer
time. Additionally, for the cakey brownies, add baking powder to improve the
height and the fluffiness .

Farthermore, using less sugar will prevent the sweetness from overwhelming the
bitter taste of the chocolate and the cocoa resulting in a richer flavour
variety (the following recipe already reflects that). However, the sugar, when
baking, carmelises at the top forming a light crust. Adding too little sugar
will result in no crust and extremely bitter brownies.

The following recipe is for a 22x22cm baking pan of fudgy brownies.

\subsection*{Ingredients}
\begin{enumerate}
  \item 180g 64\% chocolate
  \item 115g butter
  \item 110g white sugar
  \item 3 eggs
  \item 60g flour
  \item 50g cocoa powder
  \item 7g salt
  \item 120g 64\% chopped chocolate or chocolate chips
\end{enumerate}

Preheat your oven to 165C and line a 22x22cm baking pan with parchment paper.

In a water bath, melt the chocolate and the butter stirring occasionally to
combine. Once fully melted, take off the heat source, add sugar and salt, and
mix. Beat the living hell out of the eggs with a mixer (at least 10 minutes).
Pour in the chocolate mixing continously on low speed. Add the dry ingredients
(flour, cocoa powder) and the chopped chocolate or chocolate chips. Fold gently
until no flour pockets remain (we do not want to deflate the batter).

Pour the batter into the pan and bake for 18-20 minutes. Let cool in the pan on
a wire rack for several hours before cutting. Cut into 9 pieces.
