\chapter{Desserts}

\section{Tiramisu}

\header{Ingredients}
\begin{itemize}
  \item 4 eggs
  \item 50g sugar
  \item 500g mascarpone
  \item ~20 savoiardi (for a recipe, see \fullref{Savoiardi})
  \item 300ml espresso (lukewarm or cold)
  \item cocoa powder
\end{itemize}

\header{Instructions}
The coffee must not be hot as otherwise the savoiardi dipped in it will melt the
mascarpone creme.

If you follow the order of whipping, you won't have to clean your mixer.
Otherwise, make sure to not stain the whites with any yolks or mascarpone,
and to not introduce mascarpone to your yolks.
\begin{enumerate}
  \item Separate whites from yolks.
  \item Whip the whites with sugar until stiff.
  \item Whip yolks until pale.
  \item Whip the mascarpone until softened. Do not overwhip or it will turn
    into butter.
  \item Fold the yolks into the whites, then fold the mascarpone in batches
    into the mix until smooth. Do it gently and as little as possible or the
    eggs will lose all the air.
  \item Lay the first layer of savoiardi dipping them in the coffee (do not soak
   them!) just before placing them.
  \item Place a thick layer of the mascarpone.
  \item Repeat with the next layer.
  \item Refrigerate for at least 8h, preferably overnight.
  \item Dust a piece with cocoa powder before serving.
\end{enumerate}

Cocoa powder absorbs the moisture from the creme, thus dusting before serving
ensures it remains powdery and light.

\section{Panna Cotta}
Panna Cotta may be served and garnished in a variety of ways, e.g. layered in a
glass or inverted onto a plate with a side of fruits. In this recipe we will be
making a 3-layer Panna Cotta in whiskey glasses, however, the recipe for the
cream itself may be used to make any variety of Panna Cotta. For a traditional
Panna Cotta, omit the chocolate.

\header{Ingredients}
For 3 servings:

\ingredientsection{Cream}
\begin{itemize}
  \item 500ml cream 36\% or 30\%
  \item 25g sugar
  \item 7g gelatin (powder, for other kinds use the appropriate method to
  dissolve)
  \item vanilla bean (optional)
\end{itemize}

\ingredientsection{Chocolate}
\begin{itemize}
  \item 30g dark chocolate 64\%+
  \item 20g butter
  \item powdered sugar (optional)
\end{itemize}

\ingredientsection{Fruit Sauce}
\begin{itemize}
  \item 100g strawberries, peaches or any other fruits
  \item mint (optional)
  \item sugar (optional)
\end{itemize}

\ingredientsection{Garnish}
\begin{itemize}
  \item 1 large strawberry, 1 slice of peach or a part of the fruit used to
  make the sauce
  \item 2 mint leaves
\end{itemize}

\header{Instructions}
\begin{enumerate}
  \item In a small bowl, mix 3-4 spoons of cream with the gelatin.
  \item Pour the cream into a large pot and add the sugar. Set over small heat.
    If you are using vanilla or other spices, add them now.
  \item Pour in the cream mixed with the gelatin and mix until thoroughly
    incorporated.
  \item Once the temperature reaches \SI{75}{\celsius}, cut the heat and
    distribute evenly into 400ml whiskey glasses.
  \item Cover tightly with plastic foil and place in the fridge for at least 4h.
    The foil prevents the cream from developing a hard skin.
\end{enumerate}

Once the cream sets and you are ready to serve the dessert, prepare the
chocolate and the fuit sauce:
\begin{enumerate}
  \item Melt the chocolate and the butter in a water bath. Adjust the sweetness
    with the powdered sugar.
  \item Blend the fruits and the mint. You may optionally cook the fruits
    before/after blending (add the mint after cooking once cooled to preserve
    the intensity of the flavour). Frozen fruits will taste worse, thus it is
    advisable to introduce additional flavour, e.g. sour through the addition
    of lemon juice.
  \item Prepare the garnish.
  \item Remove the cover from a glass and wipe any moisture from the inside and
    the outside.
  \item Layer chocolate thinly as it has quite an intense flavour.
  \item Pour the sauce gently on top and garnish with fruit and mint.
\end{enumerate}

\section{Brownies}
There are two types of brownies - fudgy and cakey. Fudgy brownies are chewy,
gooey, moist, while cakey brownies resemble a very dense genoise. It is up to
your personal preference which type you will make.

The proportions of fats (from butter and chocolate) and flour will vary the
fudgyness of the brownies. For fudgy brownies add more fat (butter and
chocolate) and for cakey brownies add more flour. The mixing technique, the
baking time and the temperature also affect the texture of the brownies. For
fudgy brownies, barely mix the ingredients and bake shorter, while for cakey
brownies do the exact opposite - ensure the ingredients are thoroughly
incorporated, preferably using a mixer, and the brownies are baked for a longer
time. Additionally, for the cakey brownies, add baking powder to improve the
height and the fluffiness .

Farthermore, using less sugar will prevent the sweetness from overwhelming the
bitter taste of the chocolate and the cocoa resulting in a richer flavour
variety (the following recipe already reflects that). However, the sugar, when
baking, carmelises at the top forming a light crust. Adding too little sugar
will result in no crust and extremely bitter brownies.

The following recipe is for a 22x22cm baking pan of fudgy brownies.

\header{Ingredients}
\begin{enumerate}
  \item 180g 64\% chocolate
  \item 115g butter
  \item 110g white sugar
  \item 3 eggs
  \item 60g flour
  \item 50g cocoa powder
  \item 7g salt
  \item 120g 64\% chopped chocolate or chocolate chips
\end{enumerate}

Preheat your oven to 165C and line a 22x22cm baking pan with parchment paper.

In a water bath, melt the chocolate and the butter stirring occasionally to
combine. Once fully melted, take off the heat source, add sugar and salt, and
mix. Beat the living hell out of the eggs with a mixer (at least 10 minutes).
Pour in the chocolate mixing continously on low speed. Add the dry ingredients
(flour, cocoa powder) and the chopped chocolate or chocolate chips. Fold gently
until no flour pockets remain (we do not want to deflate the batter).

Pour the batter into the pan and bake for 18-20 minutes. Let cool in the pan on
a wire rack for several hours before cutting. Cut into 9 pieces.
