\chapter{Pizza}
\section{Pizza Napoletana}

\header{Ingredients}
\begin{enumerate}
  \item wheat flour (11-12\% protein or type 450)
  \item 60\%+ water (room temperature)
  \item 0.5\% fresh yeast
  \item 2\% salt
\end{enumerate}

\header{Measuring}
It is important to measure the appropriate amounts of ingredients, so that we
may divide the dough evenly without any leftover (too large pieces will result
in pizzas with a thick bottom, while too small pieces will result in smaller
pizzas). We will be dividing the dough into $280 \pm 5$g pieces. For example,
if we want to make 4 pizzas, we should make 1120g of dough. The mass of the
  entire dough is approximately the sum of the masses of the ingredients, hence
  we may calculate the amount of flour needed:
$$
mass_{flour} = mass_{dough} \times \frac{100\%}{ percent_{flour} + percent_{water} + percent_{salt} + percent_{yeast} }
$$
For a 60\% hydration dough we need $\frac{1120}{1 + 0.6 + 0.02 + 0.005} =
\frac{1120}{1.625} = 689.3$g flour.

\header{Instructions}
We mix the ingredients in a large bowl, then knead using any technique for
about 20 minutes. For high hydration dough we may instead choose the folding
technique and perform it periodically during the resting period. We leave the
dough for at least 2h to rest, but due to it being a low yeast content dough,
we may leave it for 5h or more.

After the dough rests, we prepare containers to store it. We may use small
containers for each piece of dough or large fermentation boxes (as seen in
commercial pizzerias) for multiple. Cover the containers with a thin layer of
oil. Divide the dough into $280 \pm 5$g pieces, shape into balls with a smooth
surface (extremely important) by, for instance, folding the dough into itself
and place them in the containers. If storing multiple ball in one container,
leave at least 5cm of space inbetween each pair. Store in the fridge and let
ferment for at least 8h. It is possible to store the dough in the fridge for up
to a week.

After taking out of the fridge, let rest for around 30 minutes or until it
reaches around 16C. In the meantime preheat the oven to the highest available
temperature. Carefully take one dough ball out of a container (it is important
to not push the gases out of the dough at this stage) onto flour and stretch
using any technique while being cautious to not press the gases out of the
crust. Bake until the crust turns crispy (for reference, 6-8 minutes in 250C).
