\chapter{Pizza}

\section{Pizza Dough} \label{pizza-dough}
\header{Ingredients}
\begin{enumerate}
  \item wheat flour
  \item 60\%+ water (room temperature)
  \item 2\% salt
  \item 0.5\% fresh yeast
\end{enumerate}

\header{Instructions}
\begin{enumerate}
  \item Mix water, salt and yeast until dissolved.
  \item Add the flour and mix with a sturdy spoon until no loose flour remains,
    then leave to rest for 20-30 minutes.
  \item Knead the dough until homogenous (2-5 minutes or however long you wish).
  \item Shape the dough into a smooth ball.
  \saveenum
\end{enumerate}
For "immediate" use:
\begin{enumerate}
  \contenum
  \item Leave the dough for around 4 hours, covered, to bulk rise.
  \item Portion into equally sized balls and leave to rest for at least 30
    minutes.
\end{enumerate}
For overnight storage:
\begin{enumerate}
  \contenum
  \item Portion into containers, ideally the target weight balls. The
    containers must be large enough to accommodate expansion of the dough.
  \item Move the containers into a fridge for the night.
  \item Remove from the fridge at least an hour before using. If not portioned
    into balls, do it immediately after removing from the fridge to allow the
    dough to relax.
\end{enumerate}

\section{Pizza Napoletana}
% Originating from the 19th century Naples,

The recommended weight of one portion of dough is 250g.

\header{Instructions}
\begin{enumerate}
  \item Prepare the dough. See \fullref{pizza-dough}.
  \item Prepare your preferred sauce. For example, see \fullref{sauce-marinara}.
  \item Prepare your toppings. If using fresh mozzarella, dry it lightly.
  \item Stretch the pizza dough.
  \item Spread a generous amount of sauce on the dough. Do not do it in advance
    as the might become soggy.
  \item Place the toppings, cheese first.
  \item Bake at \SI{250}{\celsius} for around 8 minutes in a home oven or at
    \SI{450}{\celsius} for 90 seconds in a wood oven (or pizza oven).
\end{enumerate}

\section{Calzone}
\section{Panzerotti}
\section{Focaccia}
